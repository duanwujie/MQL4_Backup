\section{黑客}

\subsection{两头做}





\subsection{大师足迹}


\subsubsection{保罗·都铎·琼斯}

1987年10月,世界上大部分投资者损失惨重。同一个月,保罗•都铎•琼斯掌管的都铎基金却获得62\%的收益。琼斯的出色表现是一贯的,他曾经连续5年保持三位数的增长,1992年底欧洲货币体系发生危机,琼斯数月内在外汇市场赢利十几亿美元。琼斯从做经纪人起家,1976年开始做起,第二年就赚了100多万美元佣金。1980年琼斯到纽约棉花交易所当现场交易员,几年之内赚了上千万。1984年琼斯离开交易所,创建都铎基金,从150万做起。4年后投资到他的基金的每1000美元已增值到1700多美元。到1992年底,都铎基金总额已增长到60亿美元。如果不是琼斯于1987年底停止接受新投资并开始分发利润,那么60亿是绝对打不住的。


琼斯的交易生涯并非一帆风顺。1979年他逞一时之勇,一次进单过多,结果连遇跌停板,等平单出场时资金损失达三分之二。他懊丧至极,对自己几乎完全丧失信心,差一点改行。\textbf{从那以后他开始学会控制风险,遵守原则。}


\textbf{琼斯做单每天都是新的起点,昨天赚的成为过去,今天从零开始。每个月亏损最多不能超过10\%。}顺手时,琼斯可以连续十几个月不输钱。三位数的年增长率对他来说是司空见惯。由于风险控制得法,琼斯的基金在分析判断失误的情况下仍能赢利。1992年年初,琼斯认为美国减息己到尽头,欧洲利息将下降,欧美利率差的缩小将扭转美元弱势。都铎基金因而进场买了大量美元。刚开始还较顺利,美元果然走强了几百点。但不久美国经济不振的消息频传,美元对欧洲货币大幅下跌,直至创历史最低价位。琼斯在发觉大势不对后及时砍单出场,避免了更大的损失。他同时耐心等待时机,追回损失。年底欧洲货币体系发生危机,英镑、意大利里拉等货币大跌,琼斯及时进场抛售外币,一月之内赚了数亿美元。

琼斯有一些具体的做单原则:\textit{不平均加单。一批单子进场后,市场反走说明判断可能有问题,盲目加单平均价位虽然稍好,但如果方问错了,新加的单只是错上加错。反过来讲,如果你相信方问没错,只是价位不够好,那就不必过于计较。琼斯认为,哪里进单不重要,关键是这一天你是看涨还是看跌}。新手最爱问琼斯:你是买还是卖呀? 琼斯认为他是买还是卖不应该影响旁人对市场的判断。新手也要独立思考。再一个问题就是:你从哪里开始买的? 琼斯认为这也与当天是赚还是赔无关,关键是判断涨与跌。

\textbf{琼斯认为,做单最重要的是防守而不是进攻。他每天都假设自己进的每一张单都是错的,事先设好止蚀位,这样他对最多一次会赔多少心里有数}。琼斯奉劝所有交易员不要逞强,更不能自负。要不停地怀疑你自己,怀疑你的能力,永远不要自以为了不起。\textcolor{red}{一飘飘然就完蛋}。这并不是说对自己毫无信心,信心一定要有,但适可而止。琼斯自言他对这行是越干越怕,因为他知道要保持成绩有多难。\textbf{每次大输往往都是在连续做了些漂亮单后自我感觉良好之际。}

琼斯的做单策略与众不同。他不愿意随大流,很少追势,总喜欢在转势之际赚钱。他自为最大的机会主义者。一旦他发现这种机会,便进场兜底或抛顶。错了马上就砍单,然后再试,往往是试了几次以后开始赚大钱。市场上很多人认为一味找底或顶很危险,要赚钱最好抓势的中段。琼斯十多年来却成功地抓住了不少顶和底。琼斯的理论是,跟势的人要在中段赚钱,止蚀单就得设得很远,一旦被迫砍单,损失就很大。再说市场只有15\%情况下才有势,其他时间都是横走。所以他比较喜欢做两头。

琼斯觉得外汇市场任何人都操纵不了。一般人有种错觉,以为华尔街大户能控制市场价格的变化。琼斯说,他可以进场闹腾一、两天,甚至一个星期,特别是如果时机正确,他进场后加加油,可能造成某种假象,但他一停买,市场价格就会掉下来,除非市场本身就很强劲。他打了个生动的比方:你可以在冰天雪地的阿拉斯加开一家最漂亮的夏装店,但没人买,你总归要破产。琼斯还注意听取同行的意见,特别是战绩较佳的同行。\textcolor{red}{如果自己意见和他们一致,他就多做一点,如果有很大分歧。他就观望。本来他看好某一种货币想买进,但得知某位高手在抛出时,他就耐心等待}.到有一天市场开始走平,那人说;我看该出场了,他就进场大买特买。琼斯在具体分析手段方面最推崇"艾略特波"理论。他认为自己的成功很大一部分应归功于这一周期理论。\textbf{艾略特波理论是凭借黄金分割法推算市场涨跌周期的一种分析方法},在股票和期货市场广为采用。琼斯认为外汇市场也不例外。艾略特波理论吃透后,可以帮你找到很多低风险高收益的进单机会。`


谈到成功的秘诀,琼斯认为自己的长处是超脱。\textbf{任何已经发生的事都成为过去,3秒钟之前发生的事无关紧要},关键是下一步怎么办。感情上离市场要远一点,以前的看法不对就得及时修改。思想要开放,信心要坚定。他自己虽然偏好做两头,找逆转,但都锋基金也采用了几套跟势的电脑交易系统,成绩也很不错。但琼斯认为好交易员应该比电脑系统能赚钱,因为人脑能够灵活变通,更快地适应市场的变化,以及不同市场之间的差异。


\subsubsection{外汇大师成功经验总结}
\begin{itemize}
\item 大部分炒家都是从场内交易员开始入行的,随着交易规模啝交易品种的增加逐渐走到场外,冠军炒手马丁舒华兹、鹤立鸡群的理查丹尼斯啝超人斯坦利等人虽然开始都有一段适应过程,但他们都很快踏上了成功之路。
\item 风险控制:这是每个大师操盘经验的重中之重。在期货市场上长期生存的关键就是保留资金实力,给自己留下机会,幸免在一两次交易中就耗光实力。每位大师的风险操纵原则不尽相同,有些是以技术图表为依据,大部分是以资金百分比为依据,而笔者最为认同的则是高富拿设止损位的独特方法:\textbf{止损位永远设在图表上重要的价位之外},宁可减少交易量去迁就一个安全的止损位。“炒外汇”另外,风险控制大师海特提出的回避风险原则也值得我们学习,他认为,\textbf{应幸免参与行情过于激烈的品种};当出现大的亏损时要立刻通知客户,减轻心理负担;当出现风险时,要在第一时间砍仓。选择一个行业股票时,要选两家,但不是随便找两家,应选一家最好和一家最差的!
\item 短线和长线:从大部分炒家成功的经历看,他们都有从短线向长线转变的过程。短线对于投机者来说,在分析、记忆力、反应力、心理、交易通道等方面的要求都比较高,就像乒乓球员面对快速扣球只能下意识反应,而没有时间思考,这些需要平时高强度的训练,但大部分投资者湜很少有工夫去训练的,这也是短线投资者亏损面较大的原因之一。超人斯坦利是最为经典的长线炒家,为了幸免在价格波动时自己惊慌平仓出场,不惜远离市场,持仓数月卙臸数年。
\item 成功啝天才:期货市场不存在天才,理查丹尼斯啝维克多斯柏认为,智力、学历有时会成为成功交易的障碍,切勿死要面子、勇于认错、遵守可行的交易规则啝交易系统、善于总结经验,才是成功的关键。
\item 成功的时间:绝大多数大师都是从失败开始的,短则数月,长则十年,重要的是要有坚定的信心並不断总结经验。正所谓:“学者无先,达者为师”。
\item 电脑交易系统:多位大师都提到在交易中很依赖自己设计的电脑交易系统,电脑交易系统可以幸免人为的情绪影响,特别是在市场较为混乱时,还能坚决执行既定的交易计划,使投资者保持前后一致的获胜概率。当然,每个人必须学会开发适合自己的电脑交易系统。
\item 技术分析和内幕消息:多数大师都以技术分析为入市依据,技术分析能给投资者提供准确的入市时机,这是基本面分析法不能实现的。如超人斯坦利在1974年5月买入了小麦期货合约,半年后就翻了50\%,当记者寻问他是否知道俄罗斯购买小麦的内幕消息时,斯坦利回答:“我一点都不知道俄罗斯在买小麦,但图表告诉我有人在买。”大部分投资者无法靠技术分析法获利的原因在于,他们並没有掌握技术分析法的真谛。而作为大户必须同时使用基本分析法,因为他们资金庞大,建立头寸啝离场时间较长,靠技术分析法建立头寸和离场是较为困难的。
\item 人生哲学:期货市场是遵守零啝定律的,个别人赚大钱就意味着大部分人在亏钱。因此,大师的性格是有异于常人的,他们大多孤僻而充满自信,不喜欢和别人谈论行情和消息,习惯于独立分析市场。
\end{itemize}


