%\chapter{交易策略}

%
%\item 趋势指标判断趋势,震荡指标是告诉你位置,动量指标是告诉你方向.
%\item 当建立的仓位的趋势,方向,位置一致时这个仓位是好的仓位.
%\item 当建立的仓位的趋势和方向不一致而方向和位置一直时,可以做短线或中线.
%\item 第一目标,找到趋势方向位置一致的仓位!


\begin{itemize}
\item 多重时间框架,使用1小时,4小时和日图3重时间段.
\item \textbf{货币对选择}:\textit{通过日图来判断市场趋势方向,比较不同货币选择趋势最为明显的货币对.}
\item \textbf{趋势}:\textit{通过日图200均线判断趋势方向,如果价格低于200均线,则趋势是向下的,如果价格高于200均线,则趋势是向上的.}
\item \textbf{方向}:\textit{使用4小时MACD坡度来判断方向坡度越大说明趋势越强,通过4小时MACD坡度过滤出可靠的信号,判断出方向.}
\item \textbf{进场位置}:\textit{根据RSI在合适的位置开方向一致的仓位.}
\item 根据MACD信号判断出的方向和趋势和位置,不一致时,只能做短线!
\item 当出现趋势延续性信号,并且和趋势时,可以长线进行持有.
\item \textbf{出场位置}:
\end{itemize}


日内SRDC交易方法:
\begin{itemize}
\item 基于这一事实,我们可以放心假设,每日的高点和低点都是不一样滴!
\item 画出前一日的阻力与支撑位,支撑位是蜡烛的最低价位,阻力位就是最高价位!
\item 看到下一个k线与那两条线交叉了么!是,开始交易,不是,不交易
\item 当下一个k图与支撑位交叉时做空,超过阻力位时做多。
\end{itemize}







