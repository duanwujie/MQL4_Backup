%\chapter{}
\section{4小时MACD交易策略}


%
%\item 趋势指标判断趋势,震荡指标是告诉你位置,动量指标是告诉你方向.
%\item 当建立的仓位的趋势,方向,位置一致时这个仓位是好的仓位.
%\item 当建立的仓位的趋势和方向不一致而方向和位置一直时,可以做短线或中线.
%\item 第一目标,找到趋势方向位置一致的仓位!


\begin{itemize}
\item 多重时间框架,使用1小时,4小时和日图3重时间段.
\item \textbf{货币对选择}:\textit{通过日图来判断市场趋势方向,比较不同货币选择趋势最为明显的货币对.}
\item \textbf{趋势}:\textit{通过日图200均线判断趋势方向,如果价格低于200均线,则趋势是向下的,如果价格高于200均线,则趋势是向上的.}
\item \textbf{方向}:\textit{使用4小时MACD坡度来判断方向坡度越大说明趋势越强,通过4小时MACD坡度过滤出可靠的信号,判断出方向.}
\item \textbf{进场位置}:\textit{判断出方向后,根据RSI在合适的位置开方向一致的仓位.}
\item \textbf{出场位置}: \textit{止盈有多个位置,根据8EMA,21EMA,89EMA以及200SMA来设定出场位,89EMA是个重要位置,当接近89EMA关闭\%80的仓位,当到达200EMA关闭所有仓位}
\item \textbf{暂时观望也是种策略}
\item \href{run:4Hour MACD deals.xlsm}{\textit{交易记录}} 
\end{itemize}

\section{资金管理}

攻守之道,在于先立于不败之地而后求胜! 存己才能灭敌!

假设你拥有10000美元,你亏掉了其中了5000美元。那么你的亏损占整个账户的多少百分比呢?
答案是\%50.非常简单的一个问题。现在你需要增长多少百分比才能回到原来的10000美元呢?
现在不是\%50了,你需要增长\%100才能回到以前的10000美元了。这个例子告诉我们亏损是非常容易的,但是赚钱确实困难的.

你损失的越多,则你要返回你的原始资本所需要作出的努力越大!从另一个侧面来讲,也就是你面临的困难将更大!

\begin{itemize}
\item \textbf{风险回报比}:\textit{控制在1:2到1:3之间}
\item \textbf{原则1}:\textit{每次加仓前必须要求已经开立的头寸处于盈利的情况下才能加仓}
\item \textbf{原则2}:\textit{坚持递减仓量的加仓方式。每次的加仓量都不应该大于前面的开仓仓量}
\item \textbf{原则3}:\textit{根据交易波段的长短控制加仓次数,连续加仓次数一般不超过3次}
\item \textbf{重要原则}:\textit{每单最大损失只能是账户的\%3,每个月的损失不超过整个账户的\%10}
\item \href{run:Risk Position Size Calculation.xlsx}{\textit{开仓部位计算}} 
\end{itemize}


\section{EA策略}
\begin{itemize}
\item 一个EA包括趋势判断,方向判断,位置判断,仓位管理.
\item 方向判断包括买入点的判断和卖出点的判断.
\item 趋势判断:趋势包括长期趋势,中期趋势和短期趋势.
\item 通过200日SMA均线来判断长期趋势.





\end{itemize}



%日内SRDC交易方法:
%\begin{itemize}
%\item 基于这一事实,我们可以放心假设,每日的高点和低点都是不一样滴!
%\item 画出前一日的阻力与支撑位,支撑位是蜡烛的最低价位,阻力位就是最高价位!
%\item 看到下一个k线与那两条线交叉了么!是,开始交易,不是,不交易
%\item 当下一个k图与支撑位交叉时做空,超过阻力位时做多。
%\end{itemize}



