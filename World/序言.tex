\chapter{介绍}

我有一个缺点,对于所有未知的事物我都想去知道,这导致随着我的工作的进行,就变得越来越厌倦工作了,总以为不管努不努力,我都能很好的
完成,这样工作就变得乏味起来,失去了兴趣,但当我开始了解市场时,我就迷上了,因为明天永远是未知的,市长也总是在变化,自己永远不会
有精通的一天,必须不断的适应市场,必须不断的认错,不断的冒险,不断的学习,对我来说这个行当是完美的工作,因为永远有事做,永远在变化!

注重过程而不是利润,当你在注重利润的时候,你就变的一无所有了!


本书想通过技术分析,经济数据分析和反身性理论来发展索罗斯的反身性理论,以期望解决如何作出具体预言的步骤! 

索罗斯在提出反身性的哲学时,对基本面分析和和技术分析都全盘否定,反身性理论也并没有找到好的方式来进行为来作出具体预言!
根据反身性理论在不完备的知识和参与者的偏向情况下,预言和事件一定会存在失真,因此产生了将基本分析和技术分析应用于反身性理论
以让预言和事件无限的接近!并提出一个能大概率进行预言的反身性系统与工具!

基本数据间的反身性,我们认为主流偏向是一个相对的概念,两个国家之间只会存在相对较好和相对较差两种,因此在两国之间
只会存在一个主流偏向,并随着时间的改变进行加强或修正.



